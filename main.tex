% document class
\documentclass[11pt]{amsart}

% AMS packages
\usepackage{amsmath, amsthm, amssymb}

\usepackage{parskip}%No Indent
  
% fonts
\usepackage{lmodern}
\usepackage{palatino}                       % text font
\usepackage[bigdelims,vvarbb]{newpxmath}    % math font
\usepackage[scaled=0.95]{inconsolata}       % Sans font 
\linespread{1.12}                           % spacing
\usepackage[T1]{fontenc}                    % font encoding 


% comments
\usepackage{comment}

% figures
\usepackage[abs]{overpic}		  % overpic automatically loads graphicx
\usepackage[all,cmtip]{xy}

% hyperlinks
\usepackage{xcolor}
\definecolor{indigo}{rgb}{0.29, 0.0, 0.51}  % custom colors
\usepackage[colorlinks, urlcolor=indigo, linkcolor=indigo, citecolor=indigo]{hyperref}

% margins
\usepackage[hcentering, vcentering, total={5.9in, 8.2in}]{geometry}  % 1.4 vertical margin; 1.3 horizontal margin

% tikz
\usepackage{tikz}
\usetikzlibrary{trees}

% arrows
\usepackage{mathtools}

% theorems
\theoremstyle{plain}
\newtheorem{theorem}{Theorem}
\newtheorem{corollary}[theorem]{Corollary}
\newtheorem{proposition}[theorem]{Proposition}
\newtheorem{lemma}[theorem]{Lemma}
\newtheorem{question}[theorem]{Question}
\newtheorem{problem}[theorem]{Problem}
\newtheorem{conjecture}[theorem]{Conjecture}
\newtheorem{claim}{Claim}

% definition
\theoremstyle{definition}
\newtheorem{definition}[theorem]{Definition}

% remark
\theoremstyle{remark}
\newtheorem{remark}[theorem]{Remark}
\newtheorem{example}[theorem]{Example}
\newtheorem{observation}[theorem]{Observation}
\newtheorem{construction}[theorem]{Construction}

% theorem counter
\numberwithin{theorem}{section}

% macros
% Do not use \def, use \newcommand
% basic macros
\newcommand{\dfn}[1]{{\em #1}}        % definition
\DeclareMathOperator{\bd}{\partial}   % boundary
\DeclareMathOperator{\interior}{int}  % interior
\newcommand{\modp}[1]{\;(\!\!\!\!\!\!\mod #1)}      % mod for display math. \pmod is for inline math 
\newcommand{\case}[1]{\begin{cases} #1 \end{cases}} % cases

% mathbb
\newcommand{\R}{\mathbb{R}}           % the real numbers
\newcommand{\Q}{\mathbb{Q}}           % the rational numbers
\newcommand{\Z}{\mathbb{Z}}           % the integers
\newcommand{\CC}{\mathbb{C}}           % the complex numbers
\newcommand{\N}{\mathbb{N}}           % the natural numbers
%\newcommand{\F}{\mathbb{F}}           % field
\newcommand{\NS}{{\mathbb{S}}}
\newcommand{\T}{{\mathbb{T}}}
\newcommand{\D}{{\mathbb{D}}}
\newcommand{\B}{{\mathbb{B}}}
\newcommand{\PP}{{\mathbb{P}}}
\newcommand{\Op}{{\mathcal{O}p}}
\newcommand{\ST}{{\mathcal{T}}}

% mathcal
\newcommand{\K}{\mathcal{K}}           % spaces of knots
\renewcommand{\L}{\mathcal{L}}           % spaces of Legendrian knots
\newcommand{\FL}{\mathcal{FL}}           % spaces of formal Legendrian knots
\newcommand{\KL}{\mathcal{KL}} 
\newcommand{\KC}{\mathcal{KC}} 

% bigger cdot, smaller than bullet
\makeatletter
\newcommand*\bigcdot{\mathpalette\bigcdot@{0.6}}
\newcommand*\bigcdot@[2]{\mathbin{\vcenter{\hbox{\scalebox{#2}{$\m@th#1\bullet$}}}}}
\makeatother

% vectors
\newcommand{\vect}[2]{\left(\begin{matrix} #1 \\ #2 \end{matrix}\right)}             % 2x1 vector
\newcommand{\vects}[2]{\left(\begin{smallmatrix} #1 \\ #2 \end{smallmatrix}\right)}  % inline 2x1 vector
\newcommand{\dvec}[1]{\frac{\partial}{\partial #1}}                                % differential vector
\newcommand{\dvecin}[1]{\partial / \partial #1}                                % inline differential vector

% matrices
\newcommand{\matrixs}[4]{{\bigl(\begin{smallmatrix}#1&#2\\ #3&#4\end{smallmatrix} \bigr)}}  % inline 2x2 matrix
\newcommand{\matrixp}[4]{{\begin{pmatrix}#1 & #2 \\ #3 & #4\end{pmatrix}}}                  % parenthesis 2x2 matrix
\newcommand{\matrixb}[4]{{\begin{bmatrix}#1 & #2 \\ #3 & #4\end{bmatrix}}}                  % bracket 2x2 matrix

% contact geometry macros
%\DeclareMathOperator{\Tight}{Tight}
%\DeclareMathOperator\tb{tb}                               % Thurston-Bennequin
%\DeclareMathOperator\tbb{\overline {\tb}}                 % maximum Thurston-Bennequin
%\DeclareMathOperator\rot{rot}                             % rotation
%\DeclareMathOperator\self{sl}                             % self linking
%\DeclareMathOperator\selfb{\overline {\self}}             % maximum self linking
%\DeclareMathOperator\tw{tw}                               % twisting number
\DeclareMathOperator\twb{\overline {\tw}}                 % maximum twisting number

% gauge theory macros
\DeclareMathOperator{\SW}{\mathcal{SW}}     %SW
\DeclareMathOperator{\spins}{\mathfrak{s}}
\DeclareMathOperator{\spint}{\mathfrak{t}}
\DeclareMathOperator{\relspins}{\underline{\mathfrak{s}}}
\DeclareMathOperator{\relspint}{\underline{\mathfrak{t}}}
\DeclareMathOperator{\spinc}{\mathrm{Spin}^{\textit{c}}}  %Spin^c
\DeclareMathOperator{\relspinc}{\underline{\mathrm{Spin}^{\textit{c}}}}

% Heegaard Floer macros
\DeclareMathOperator{\HFhat}{\widehat{{HF}}}  % HF^
\DeclareMathOperator{\HFplus}{{{HF}}^+}       % HF+
\DeclareMathOperator{\HFminus}{{{HF}}^{--}}   % HF-
\DeclareMathOperator{\CFhat}{\widehat{CF}}    % CF^    
\DeclareMathOperator{\CFplus}{{{CF}}^+}       % CF+    
\DeclareMathOperator{\CFminus}{{{CF}}^{--}}   % CF-    
\DeclareMathOperator{\HFKhat}{\widehat{HFK}}  % HFK^
\DeclareMathOperator{\HFKminus}{{{HFK}}^{--}} % HFK-
\DeclareMathOperator{\CFKhat}{\widehat{CFK}}  % CFK^
\DeclareMathOperator{\CFKminus}{{{CFK}}^{--}} % CFK-

% 4-manifolds macros
\DeclareMathOperator{\cp}{\mathbb{C}{{P}}^2}  % CP2
\DeclareMathOperator{\cpbar}{\overline{\mathbb{C}P^2}}  % CP2bar

% diffeomorphism macros
%\DeclareMathOperator{\Diff}{Diff}         % group of diffeomorphisms 
%\DeclareMathOperator{\Cont}{Cont}         % group of contactomorphisms 
\DeclareMathOperator{\PB}{{{PB}}}          % pure braid group 

%Import symbols from font cmr without importing the whole package
\DeclareFontFamily{U} {cmr}{}
\DeclareFontShape{U}{cmr}{m}{n}{
  <-6> cmr5
  <6-7> cmr6
  <7-8> cmr7
  <8-9> cmr8
  <9-10> cmr9
  <10-12> cmr8
  <12-> cmr9}{}
\DeclareSymbolFont{Xcmr} {U} {cmr}{m}{n}
\DeclareMathSymbol{\Phi}{\mathord}{Xcmr}{8}

\DeclareMathOperator\tbb{\overline {\tb}}     % maximum Thurston-Bennequin
\newcommand{\tcan}{\mathbb{T}^2_{\text{can}}}
\newcommand{\norm}[1]{\left\lVert#1\right\rVert}



\newcommand{\Gr}{\operatorname{Gr}}
\newcommand{\GL}{\operatorname{GL}}
\newcommand{\SL}{\operatorname{SL}}
\newcommand{\PGL}{\operatorname{\mathbb{P}GL}}
\newcommand{\Diff}{\operatorname{Diff}}
\newcommand{\Ort}{\operatorname{O}}
\newcommand{\SO}{\operatorname{SO}}
\newcommand{\U}{\operatorname{U}}
\newcommand{\Sp}{\operatorname{Sp}}
\newcommand{\SU}{\operatorname{SU}}
\newcommand{\End}{{\operatorname{End}}}
\newcommand{\Mon}{\operatorname{Mon}}
\newcommand{\Ar}{\operatorname{Area}}
\newcommand{\Int}{\operatorname{Int}}
\newcommand{\Mono}{\operatorname{Mono}}
\newcommand{\Iso}{\operatorname{Iso}}
\newcommand{\rel}{\operatorname{rel}}
\newcommand{\real}{\operatorname{real}}
\newcommand{\SHE}{\operatorname{SHE}}
\newcommand{\RR}{\operatorname{ref}}
\newcommand{\dist}{\operatorname{dist}}
\newcommand{\Maps}{\operatorname{Maps}}



\newcommand{\im}{{\operatorname{Image}}}
\newcommand{\Id}{{\operatorname{Id}}}
\newcommand{\fr}{\operatorname{fr}}
\newcommand{\rot}{\operatorname{rot}}
\newcommand{\tb}{\operatorname{tb}}
\newcommand{\rotconr}{\operatorname{r}}
\newcommand{\tw}{\operatorname{tw}}
\newcommand{\Span}{\operatorname{span}}
\newcommand{\Sl}{\operatorname{sl}}

%\newcommand{\Top}{\operatorname{Top}}
\newcommand{\Cont}{\operatorname{Cont}}
\newcommand{\FCont}{\operatorname{FCont}}
\newcommand{\HM}{\operatorname{H}}%HOMOLOGIA/COHOMOLOGIA
\newcommand{\Fr}{\operatorname{Fr}}
\newcommand{\CFr}{\operatorname{CFr}}
\newcommand{\ev}{\operatorname{ev}}
\newcommand{\Symp}{\operatorname{Symp}}
\newcommand{\Tight}{\operatorname{Tight}}
\newcommand{\SCont}{\operatorname{CStr}}
\newcommand{\Funct}{\operatorname{{\mathcal F} unct}}
\newcommand{\Good}{\operatorname{Good}}
\newcommand{\Germs}{\operatorname{Germs}}

\newcommand{\pr}{\operatorname{pr}}

\newcommand{\image}{\operatorname{Image}}

\newcommand{\C}{\operatorname{C}}
\newcommand{\FC}{\operatorname{FC}}
\newcommand{\OT}{\operatorname{OT}}
\newcommand{\SOT}{\operatorname{SOT}}
\newcommand{\Emb}{{\operatorname{Emb}}}
\newcommand{\FEmb}{{\operatorname{FEmb}}}
\newcommand{\FDiff}{{\operatorname{FDiff}}}
\newcommand{\F}{{\operatorname{\mathcal{F}}}}
\newcommand{\Germ}{\operatorname{Germ}}
\newcommand{\support}{\operatorname{Support}}



%*********************************************************************
\begin{document} 

% title
\title{Contact anti-surgery and Legendrian Rigidity in overtwisted contact manifolds}

\author{Eduardo Fern\'andez}

\author{Fabio Gironella}


\maketitle

\begin{abstract}
    We use contact anti-surgeries to find new instances of rigidity phenomena for Legendrian embeddings. 
    On one hand, in each dimension $2n+1\geq 5$, we show the existence of infinitely many overtwisted contact $(2n+1)$-manifolds for each of which the formal class of the standard Legendrian unknot admits infinitely many Legendrian representatives, up to Legendrian isotopy. 
    These examples include every homotopically standard overtwisted sphere. 
    On the other hand, in every dimension, we prove that every contactomorphism formally isotopic to the identity, relative to a Darboux ball, is generated by a $1$-parameter Legendrian surgery operation along a suitable Legendrian loop in some overtwisted contact manifold. 
    As a corollary, we show that in every overtwisted contact $3$-manifold there are infinitely many loops of loose Legendrians which are contractible among formal Legendrians but non-contractible among Legendrians.
\end{abstract}

\textcolor{red}{to think: contact pushoffs of the legendrians, diagram for vogel example, surgery realization in higher dimensions (this is about the algebraic topology really, difference between almost contact and formal contact), the non-loose unknots are formally standard, add comparison with 4D setting where they look at pseudoisotopies as motivation for last thm.}

\textcolor{red}{to write/edit: add some pictures:):) all the discussion about non-loose legendrians in Athens, preliminary stuff about Legendrian surgeries/cancellation, add Weinstein tubular neighborhood with parameters for the surgery HOMOMORPHISM. Finish the proof of Lemma 2.1. (formal isotopy fixing a ball)}

\section{Introduction}

\subsection{Non-loose Legendrians in overtwisted contact manifolds}

The main goal is to prove the following: 

\begin{theorem}\label{thm:NonLoose}
Let $(M^{2n+1},\xi)$, $2n+1\geq 5$, be an almost Weinstein fillable overtwisted contact manifold with $c_1(M,\xi)=0$. Then, there exists an infinite family of Legendrian spheres $\Lambda_i\subseteq (M,\xi)$, $i\in \Z_{\geq0}$, such that
\begin{itemize}
    \item [(a)] $\Lambda_i$ is non-loose and, in particular, $(M\backslash \Lambda_i,\xi)$ is tight, 
    
    \item [(b)] $\Lambda_i$ is formally Legendrian isotopic to the standard Legendrian unknot.
    
    \item [(c)] $\Lambda_i$ and $\Lambda_j$ are not Legendrian isotopic if $i\neq j$. 
    
\end{itemize}
\end{theorem}

\begin{remark}
    The dimensional restriction in Theorem \ref{thm:NonLoose} is essential. In dimension $3$, every Legendrian unknot formally Legendrian equivalent to the standard Legendrian unknot is actually Legendrian isotopic to it. In particular, when the ambient contact $3$-manifold is overtwisted then is loose. 
\end{remark}

\subsection{Formally contractible but non-contractible loops of Legendrians in overtwisted contact $3$-manifolds}

Let $(M^3,\xi)$ be a contact $3$-manifold and $\Lambda\subseteq (M,\xi)$ an embedded Legendrian link. We will denote by $\L(\Lambda,(M,\xi))$ the path-connected component of the inclusion $\Lambda\hookrightarrow (M,\xi)$ in the space of Legendrian embeddings into $(M,\xi)$. Similarly, we will denote by $\FL(\Lambda,(M,\xi))$ the path-connected component of the inclusion $\Lambda\hookrightarrow (M,\xi)$ in the space of formal Legendrian embeddings. 

We are interested in the group 

$$ \KL_1(\Lambda,(M,\xi)):=\ker ( \pi_1(\L(\Lambda,(M,\xi)))\rightarrow \pi_1(\FL(\Lambda,(M,\xi))) ) $$

\begin{theorem}\label{thm:LoopsOfLegendrians}
Let $(M^3,\xi)$ be an overtwisted contact $3$-manifold. Then, there exists an infinite family of Legendrians $\Lambda_i\subseteq (M,\xi)$, $i\in \Z_{\geq0}$ such that
\begin{itemize}
    \item [(a)] $\Lambda_i$ is loose,
    \item [(b)] $\Lambda_i$ is not smoothly isotopic to $\Lambda_j$ for $i\neq j$,
    \item [(c)] The group $\KL_1(\Lambda_i,(M,\xi))$ is non-trivial.
\end{itemize}
\end{theorem}
\textcolor{red}{to think about: high dimension examples}

The previous result follows from a general construction that we describe below. Given a Legendrian $\Lambda\subseteq (M,\xi)$ we will denote by $(M(\Lambda),\xi(\Lambda))$ the Legendrian surgery of $(M,\xi)$ along $\Lambda$. 

Let $\Lambda^\theta\subseteq (M,\xi)$, $\theta\in \NS^1$, be a loop of Legendrian \em embeddings \em with base point $\Lambda=\Lambda^0$. Out of this data we can produce a contactomorphism of $(M(\Lambda),\xi(\Lambda))$ as follows. Perform a $1$-parameter family of Legendrian surgeries along $\Lambda$ to build a bundle 
$$ (M(\Lambda),\xi(\Lambda))\hookrightarrow X^{2n}\rightarrow \NS^1 $$
where the fiber over $\theta\in \NS^1$ is $(M(\Lambda^{\theta}),\xi(\Lambda^{\theta}))$. The monodromy of this bundle is naturally a contactomorphism 
$$ W(\Lambda^\theta)\in \Cont(M(\Lambda),\xi(\Lambda)). $$
We will refer to $W(\Lambda^\theta)$ as \em Legendrian surgery contactomorphism (of $\Lambda^\theta$).\em

In fact, as a consequence of of the contractibility of the space of Weinstein neighborhoods of a given Legendrian \textcolor{red}{Write Lemma}, this assignment defines a homomorphism
$$ W: \pi_1(\L(L,(M,\xi)))\rightarrow \pi_0(\Cont(M(\Lambda),\xi(\Lambda))). $$

It is natural to ask which contactomorphisms of a given contact manifold $(M,\xi)$ can be realized as Legendrian surgery contactomorphisms. We prove that every potentially ``geometrically interesting'' contactomorphisms arise in this way. Here, by potentially ``geometrically interesting'' we mean elements in the group 
$$ \KC_0 (M,\xi)=\ker( \pi_0(\Cont(M,\xi))\rightarrow \pi_0(\FCont(M,\xi))). $$

The general realization result reads as follows

\begin{theorem}\label{thm:SurgeryContactomorphism}
    Let $(N,\xi_{\OT})$ be an overtwisted contact manifold and $\Lambda\subseteq (N,\xi_{\OT})$ a Legendrian such that $(N(\Lambda),\xi_{\OT}(\Lambda))=(\NS^{2n+1},\xi_{std})$. Then, for every contact manifold $(M,\xi)$ the restriction homomorphism 
    $$ W: \KL_1(\Lambda, (M,\xi)\#(N,\xi_{\OT}))\rightarrow \KC_0(M,\xi) $$
    is surjective. 
\end{theorem}






% \begin{theorem}\label{thm:main}
% Let $(M^{2n+1},\eta)$, $2n+1\geq 7$, be an almost Weinstein fillable almost contact manifold. Then, there exists an overtwisted contact structure $\xi$ on $M$ in the same almost contact class as $\eta$ and an infinite family of Legendrian spheres $\Lambda_i\subseteq (M,\xi)$, $i\in \Z_{\geq 0}$, such that
% \begin{itemize}
%     \item [(a)] $(M\backslash \Lambda_i,\xi)$ is tight,
%     \item [(b)] $\Lambda_i$ and $\Lambda_j$ are formally Legendrian isotopic for all $i,j$,
%     \item [(c)] $\Lambda_i$ and $\Lambda_j$ are not Legendrian isotopic if $i\neq j$.
% \end{itemize}
% \end{theorem}



% \textcolor{red}{E:The next is very cofusing badly written. It is just to not foget about it} Maybe a more abstract/useful statement is the following one. Let $\FC(M;\rel B)$ be the space of formal contact structures on $M$ that coincide over a Darboux ball $B\subseteq M$ and $C(M;\rel B)\hookrightarrow \FC(M;\rel B)$ the subspace of contact structures. Denote by $$K\C_{j}^{(M,\xi)}=\ker(\pi_j(\C(M,\rel B),\xi))\rightarrow \pi_j(\FC(M,\rel B,\xi))).$$

% Given a contact manifold $(M,\xi)$ we denote by $\FL(M,\xi)$ and $\L(M,\xi)$ the space of formal Legendrian embeddings and Legendrian embeddings of spheres into $(M,\xi)$. Denote by 
% $$ K\L_{j}^{L,(M,\xi)}=\ker(\pi_j(\L(M,\xi),L)\rightarrow \pi_j(\FL(M,\xi),L)) $$

% \begin{theorem}
%     Let $\Lambda\subseteq (B,\xi_{std})$ the standard Loose Legendrian unknot. Assume that there exists a non-zero element $A\in K\C_{j}^{(M,\xi)}$. Then, there exists a non-zero element $\tilde{A}\in K\L_{j}^{L,(M(+\Lambda),\xi_{+\Lambda})}.$ Here, $(M(+\Lambda),\xi_{+\Lambda})$ is obtained from $(M,\xi)$ from Legendrian surgery along $\Lambda$ and $L$ is the resulting belt Legendrian. 
% \end{theorem}

\section{Formal contactomorphisms}

\textcolor{red}{The following lemma is not essential but makes the statement about Legendrian surgery contactomorphisms more appealing since it avoids asking the relative to a ball in the statement}


\begin{lemma}\label{lem:FormallyTrivialRelativeBall}
    Let $(M,\xi)$ be a contact manifold, $\varphi\in \Cont(M,\xi)$ a contactomorphism formally isotopic to the identity and $\B\subseteq (M,\xi)$ a Darboux ball. Then, $\varphi$ is contact isotopic to a contactomorphism $\hat{\varphi}$ such that
    \begin{itemize}
        \item $\hat{\varphi}_{\B}=\Id$.
        \item $\hat{\varphi}$ is formally contact isotopic to the identity relative to $\B$. 
    \end{itemize}
\end{lemma}
\begin{proof}
    Proof idea, details to be filled. Consider a formal isotopy to the identity $(\varphi_t,F_{s,t})$. 
    \begin{itemize}
        \item \underline{Step I:} First after a homotopy make $F_{s,t}(\xi_{|\B})$ contact on $\varphi_t(\B)$ for all $s,t$. 
        \item \underline{ Step II:} Let $p\in \B$ be the center of the ball. After a Hamiltonian isotopy we can assume $\varphi_t(p)=p$. 
         \item \underline{ Step III:} Consider the family of frames $(p, F_{s,t}(p))$, by smooth isotopy extension (or by hand) we can find diffeos $H_{s,t}\in \Diff(M)$ such that 
    \begin{itemize}
        \item $H_{s,t}=\Id$ for $(s,t)\in [0,1]\times \{0,1\}\cup \{0\}\times [0,1]$
        \item $(H_{s,t})_{*} F_{1-s}(p)=F_1(p)$.
    \end{itemize}
    With this diffeos we can homotope $(\varphi_t,F_{s,t})$ into a new homotopy $(\hat{\varphi}_t,\hat{F}_{s,t})$ that is actually contact on $p$: $\hat{F}_{s,t}(p)=d\hat{\varphi}_t$.
    \item \underline{Step  IV:} Now we use Moser/Gray and the hypothesis of Step I plus the last condition to make all the $\hat{\varphi}_t$ contact near $p$. 
    \item \underline{Step V:} Since all the $\hat{\varphi}_t$ are contact near $p$ (in a ball $\B$) we can generate the path of \textbf{Darboux balls} $\hat{\varphi}_t(\B)$ by a Hamiltonian isotopy. We use this Hamiltonian isotopy to finally fix the ball. 
    \end{itemize}
\end{proof}

\begin{lemma}\label{lem:FormalContact3D}
Let $(M,\xi)$ be an overtwisted contact $3$-manifold and $\Delta_{\OT}\subseteq (M,\xi)$ an overtwisted disk. Then, the inclusion $\Cont(M,\xi;\rel \Delta_{\OT})\hookrightarrow \FCont(M,\xi;\rel \Delta_{\OT})$.
\end{lemma}
\begin{remark}
    \textcolor{red}{E:In the higher dimensional case: almost contactomorphisms. We need to think if this is enough for the Legendrian surgery realization statement}
\end{remark}

\section{Legendrian surgery}

\begin{lemma}[Cancellation Lemma, Ding-Geiges] \label{lem:Cancellation}
Let $(M,\xi)$ be a contact manifold and $\Lambda\subseteq (M,\xi)$ a Legendrian sphere. Consider a positive push-off $\hat{\Lambda}$ of $\Lambda$. Then, $(M(\pm\Lambda)(\mp\hat{\Lambda}),\xi(\pm\Lambda)(\mp\hat{\Lambda}))\cong (M,\xi)$
\end{lemma}

\section{The group of overtwisted $3$-spheres}

Overtwisted contact structures on the $3$-sphere were classified by Eliashberg as a consequence of the $h$-principle for overtwisted contact structures. In particular, they are classified by their formal data. 

Let us recall the classification. Fix the quaternionic trivialization of the tangent bundle $$\NS^3\times \R^3\cong T\NS^3,(p,(v_1,v_2,v_3))\mapsto (p, v_1 i\cdotp +v_2 j\cdotp +v_3 k\cdot p).$$ With this coordinate system we can consider the Gauss map of every co-oriented contact structure on $\NS^3$ to define a map $$\C(\NS^3)\rightarrow \Maps(\NS^3,\NS^2),\xi\mapsto f_\xi.$$ Notice that $f_{\xi_{std}}\equiv (1,0,0)$ is the constant map. We denote the Hopf invariant of $f_\xi$ as $h(f_\xi)$.

\begin{theorem}
    $\pi_0(\C_{\OT}(\NS^3))=\{\xi_k:k\in \Z\}$ where $\xi_k$ is defined as the unique (up to isotopy) overtwisted contact structure such that $h(f_{\xi_k})=k$.
\end{theorem}

With this trivialization the Hopf invariant is additive under contact connected sum. In particular, 

\begin{corollary}
    The monoid morphism $(\pi_0(\C_{\OT}(\NS^3)),\#)\rightarrow (\Z,+), \xi_k\mapsto k;$ is a group homomorphism. 
\end{corollary}

\begin{corollary}\label{cor:ConnectedSumOTSphere}
    Let $(M,\xi)$ be an overtwisted contact $3$-manifold. Then, for every $k\in \Z$, there exists an overtwisted contact structure $\hat{\xi}$ on $M$ such that $(\NS^3,\xi_k)\#(M,\hat{\xi})\cong (M,\xi)$.
\end{corollary}
\begin{proof}
    Take $(M,\hat{\xi})=(M,\xi)\#(\NS^3,\xi_{-k})$.
\end{proof}

\section{Legendrian surgery on overtwisted contact $3$-manifolds}

\begin{theorem}[Etnyre-Honda]\label{thm:OTLegendrianSurgery}
    Let $(M,\xi)$ be an overtwisted contact structure. Then, there exists a infinitely many Legendrian links $\Lambda_i\subseteq (M,\xi)$, $i\in \Z_{\geq 0}$, such that $(M(\Lambda_i),\xi(\Lambda_i))\cong (\NS^3,\xi_{std})$.
\end{theorem}
\begin{proof}
    Etnyre-Honda (check also Geiges-Ding-Stipsicz) found a Legendrian link $\Lambda\subseteq (M,\xi)$ such that $(M(\xi),\xi(\Lambda))\cong (\NS^3,\xi_{std})$. 
    To obtain infinitely many examples we proceed as follows. Consider a Legendrian link $\Lambda_0\subseteq (\NS^3,\xi_0)$ such that $(\NS^3(\Lambda_0),\xi(\Lambda_0))=(\NS^3,\xi_{std})$. By the overtwisted  $h$-principle we can realize $(M,\xi)\cong(M,\xi)\#(\NS^3,\xi_0)$. Hence, there is a new link $\hat{\Lambda}=\Lambda\cup \Lambda_0\subseteq (M,\xi)$ such that
   Therefore, $$(M(\hat{\Lambda}),\xi(\hat{\Lambda})\cong (M(\Lambda),\xi(\Lambda))\#(\NS^3(\Lambda_0),\xi(\Lambda_0))\cong (\NS^3,\xi_{std})\#(\NS^3,\xi_{std})\cong(\NS^3,\xi_{std}).$$
   Iterating the previous process we find the infinitely many examples. 
\end{proof}

\begin{corollary}\label{cor:LooseSurgeryOTSpheres}
    Let $(M,\xi)$ be an overtwisted contact structure. Then, for every overtwisted contact $3$-sphere $(\NS^3,\xi_k)$ there exists infinitely many loose Legendrians $\Lambda_i\subseteq (M,\xi)$, $i\in \Z_{\geq 0}$, such that $(M(\Lambda_i),\xi(\Lambda_i))\cong (\NS^3,\xi_k)$.
\end{corollary}
\begin{proof}
  By Corollary \ref{cor:ConnectedSumOTSphere} we can write $(M,\xi)\cong (M,\hat{\xi})\#(\NS^3,\xi_k)$ for some overtwisted contact structure $\hat{\xi}$ on $M$. Theorem \ref{thm:OTLegendrianSurgery} implies the existence of infinitely many Legendrians $\Lambda_i\subseteq (M,\hat{\xi})$, $i\in \Z_{\geq 0}$, such that $(M(\Lambda_i),\hat{\xi}(\Lambda_i))\cong (\NS^3,\xi_{std})$. We can naturally realize this Legendrians as a family of loose Legendrians in $(M,\xi)=(M,\hat{\xi})\#(\NS^3,\xi_k)$ that we still denote by $\Lambda_i$. It follows that 
  $$(M(\Lambda_i),\xi(\Lambda_i))=(M(\Lambda_i),\hat{\xi}(\Lambda_i))\#(\NS^3,\xi_k)\cong(\NS^3,\xi_{std})\#(\NS^3,\xi_k)\cong(\NS^3,\xi_k)$$
  as required.
\end{proof}


\section{Proof of Theorem \ref{thm:SurgeryContactomorphism}}

\textcolor{red}{E:We give a proof in the $3D$-case}

Let $\varphi\in \KC_0(M,\xi)$ be a formally trivial contactomorphism. Consider any formal isotopy $(\varphi_t,F_{s,t})\in\FCont(M,\xi)$, $t\in[0,1]$, between $\Id=(\varphi_0,F_{s,0})$ and $\varphi=(\varphi_1,F_{s,1})$. By Lemma \ref{lem:FormallyTrivialRelativeBall}, after a possibly contact isotopy of $\varphi$, we may assume that every formal contactomorphism $(\varphi_t,F_{s,t})$ is the identity over a small Darboux ball $\B^3\subseteq (M,\xi)$.


Let $(N,\xi_{\OT})$ and $\Lambda\subseteq (N,\xi_{\OT})$ be an overtwisted contact $3$-manifold together with a Legendrian link such that $$(N(\Lambda),\xi_{\OT}(\Lambda))\cong (\NS^3,\xi_{std}).$$

By the Cancellation Lemma \ref{lem:Cancellation} we can find a Legendrian $\hat{\Lambda}\subseteq (\NS^3,\xi_{std})$ such that $+$-surgery on it cancels out the previous surgery, i.e.
$$(\NS^3(+\hat{\Lambda}),\xi_{std}(+\hat{\Lambda}))\cong (N,\xi_{\OT}).$$

Regarding $\hat{\Lambda}$ as a Legendrian in a Darboux ball we can also realize it as a Legendrian in $(M,\xi)$ lying in the Darboux ball $\B^3$ in which the formal isotopy $(\varphi_t,F_{s,t})$ is the identity. 

Perform a $+$-surgery on $\hat{\Lambda}\subseteq \B\subseteq (M,\xi)$ to obtain 

$$ (M(+\hat{\Lambda}),\xi(+\hat{\Lambda}))\cong(M,\xi)\#(\NS^3(+\hat{\Lambda}),\xi_{std}(+\hat{\Lambda}))\cong(M,\xi)\#(N,\xi_{\OT}). $$

Notice that Legendrian surgery on $\Lambda\subseteq (N,\xi_{\OT})\subseteq (M,\xi)\#(N,\xi_{\OT})$ bring us back to $(M,\xi)$. 

Since every formal contactomorphism $(\varphi_t,F_{s,t})$, $t\in [0,1]$, is the identity over $\B$ we may extend them as the identity over the surgery region to obtain an isotopy of formal contactomorphisms 
$$ (\hat{\varphi}_t,\hat{F}_{s,t})\in \FCont(M\#N,\xi\#\xi_{\OT}),t\in[0,1], $$
between the identity and a genuine contactomorphism $\hat{\varphi}$ that agrees with $\varphi$ away from the surgery region. Moreover, all these formal contactomorphisms are the identity over the overtwisted region $(N,\xi_{\OT})$ on the contact connected sum. 

In particular, we can find an overtwisted disk $\Delta_{\OT}$ on which every formal contactomorphism $(\hat{\varphi}_t,\hat{F}_{s,t})$ is contact and agrees to the identity. 

Apply the $h$-principle from Lemma \ref{lem:FormalContact3D} to find a homotopy of 

$$ (\hat{\varphi}_{t,u},\hat{F}_{s,t,u}), u\in[0,1], $$

such that 

\begin{itemize}
    \item $\Id=(\hat{\varphi}_{0,u},\hat{F}_{s,0,u})$ for all $u\in[0,1]$, 
    \item $\varphi=(\hat{\varphi}_{1,u},\hat{F}_{s,1,u})$ for all $u\in[0,1]$, 
    \item $(\hat{\varphi}_{t,u},\hat{F}_{s,t,u})=(\hat{\varphi}_{t},\hat{F}_{s,t})$ for $(t,u)\in \{0,1\}\times[0,1]\cup [0,1]\times \{0\}$.
    \item $(\hat{\varphi}_{t,u},\hat{F}_{s,t,u})=(\hat{\varphi}_{t,u},d\hat{\varphi}_{t,u})\in\Cont(M\#N,\xi\#\xi_{\OT})$ is contact for all $(t,u)\in \{0,1\}\times[0,1]\cup [0,1]\times\{1\}$.
\end{itemize}

We  make an abuse of notation and denote the inclusion of the Legendrina $\Lambda$ also by $\Lambda$. Consider the loop of Legendrian embeddings $\Lambda^t\subseteq (M,\xi)\#(N,\xi_{\OT})$ given by 

$$  \hat{\varphi}_{t,1}\circ \Lambda, t\in[0,1]  $$

We observe that this loop is formally trivial by construction, indeed, 

$$ (\hat{\varphi}_{t,u},\hat{F}_{s,t,u})\circ \Lambda, (t,u)\in[0,1]\times[0,1], $$

defines a formal contraction since $(\hat{\varphi}_{t,0},\hat{F}_{s,t,0})\circ \Lambda=\Lambda$ by the assumption to be the identity in the overtwisted region $(N,\xi_{\OT})$.

Moreover, by construction the Legendrian surgery contactomorphism associated to $\Lambda^t$ is 
$$ W(\Lambda^t)=\varphi\in \Cont(M,\xi). $$

Indeed, realize that the contact isotopy $\hat{\varphi}_{t,1}$ generates the loop $\Lambda^t$. Moreover, the contactomorphism $W(\Lambda^t)$ is defined to be equal to $\hat{\varphi}_{t,1}$ in the region $(M,\xi)\#(N,\xi_{\OT})\backslash \Op(\Lambda)\subseteq (M,\xi)$, so it coincides with $\varphi$ on that region; and to be the identity over the region $(\Op(\hat{\Lambda}),\xi_{std})\subseteq (M,\xi)$ on which it also coincides with $\varphi$.

This proves that the restricted Legendrian surgery homomorphism 

$$ W: \KL_1(\Lambda, (M,\xi)\#(N,\xi_{\OT}))\rightarrow \KC_0(M,\xi) $$

is surjective, as required. 

\textcolor{red}{Being precise it remains to prove the following to make sense of the statement:}

$$ W( \KL_1(\Lambda, (M,\xi)\#(N,\xi_{\OT})) ) \subseteq \KC_0(M,\xi) $$

\textcolor{red}{We should add this in a Lemma elsewhere, I haven't thought about the proof yet but it should be easy:) If it is not easy/not true we really do not care... It just looks prettier in the statement:)}

\section{Proof of Theorem \ref{thm:LoopsOfLegendrians}}

Consider the Chekanov-Vogel contactomorphism $\varphi_{CV}\in \Cont(\NS^3,\xi_{\OT})$ which is formally trivial but non-trivial. 

Let $(M,\xi)$ be any overtwisted contact $3$-manifold. By Corollary \ref{cor:LooseSurgeryOTSpheres} ther exists infinitely many loose Legendrian links $\Lambda_i\subseteq (M,\xi)$, all of them non-isotopic (smoothly), such that 
$$ (M(\Lambda_i),\xi(\Lambda_i))\cong(\NS^3,\xi_{\OT}). $$

Apply Theorem \ref{thm:SurgeryContactomorphism}  to each $\Lambda_i$ to find a formally contractible Legendrian loop $\Lambda_i^\theta\in \KL_1(\Lambda_i,(M,\xi))$, $\theta\in \NS^1$, such that the associated Legendrian surgery contactomorphism is (up to contact isotopy)
$$ W(\Lambda_i^\theta)=\varphi_{CV}\in \Cont(\NS^3,\xi_{OT})$$

Since the contactomorphism $\varphi_{CV}$ is non-trivial and $W$ is a group homomorphism we conclude that each loop $\Lambda_i^\theta$ is non-trivial. This concludes the proof.
















\bibliographystyle{plain}
\bibliography{main}
\end{document}
